\section{Research Experience}
  \resumeSubHeadingListStart

    \resumeSubheading
      {Graduate Student Researcher}{April 2024 -- now}
      {University of California San Diego}{}
      \resumeItemListStart
        \resumeItem{Worked with Michael Coblenz on two projects: a usability analysis of Copilot with a paper targeting ICSE 2026, and a separate project leveraging LLMs to assist novice programmers in detecting security risks.}
        \resumeItem{Designed an experiment using an eye tracker to evaluate the costs and benefits of Copilot for programmers new to React. Piloted it with 5 programmers, incorporating iterative improvements. The full study is undergoing.}
        \resumeItem{Designed and implemented a more universal eye-tracking platform for native VSCode to collect user interaction data across in-editor events and developed a VSCode extension to assist with the experiment.}
        \resumeItem{Developed VSCode extensions utilizing prompt-engineered GPTs that integrates static security analysis, user background, and project structure as inputs, producing two visual interfaces to assist novice programmers in detecting and understanding OWASP risks in code. The extension is undergoing iterative internal design.}
      \resumeItemListEnd

    \resumeSubheading
      {Graduate Student Researcher}{April 2024 -- now}
      {University of California San Diego}{}
      \resumeItemListStart
        \resumeItem{Worked with Jude Abishek Rayan under the guidance of Steven Dow, and collaborting with 4 others on leveraging LLM in helping group communication by inspiring people in generating more diversified, in depth ideas for CueHub project and workshop management project.}
        \resumeItem{Redesigned and deployed an online meeting platform for the CueHub project, building on a previous team's work. Implemented separate interfaces for facilitators and participants, integrated Firebase for real-time communication, Daily.js for meetings, Assembly AI for transcription and data analysis, and GPT-based cues. Resolved critical bugs and iteratively refined the platform based on pilot feedback, enabling 86 meetings.}
        \resumeItem{Co-authored on a paper submitted to CSCW 2026. Iteratively designed the platform. Designed and performed data analysis and wrote it. Wrote design decisions and implementation sections in the paper.}
        \resumeItem{For workshop management project, designed and developed a 4-component workshop management system for real-time user engagement with socket.io, electron, DDNS, MailJet and prompt-engineered GPTs to foster a better, more engaging conversation experience.}
      \resumeItemListEnd

    \resumeSubheading
      {Graduate Student Researcher}{April 2024 -- now}
      {University of California San Diego}{}
      \resumeItemListStart
        \resumeItem{Worked with Kristen Vaccaro and Deepak Kumar, and collaborated with three other group members to assess political polarization on YouTube shorts and longs. The paper is planned for submission to CSCW 2026.}
        \resumeItem{Developed a high-accuracy algorithm in JavaScript and Python (using Hugging Face) to detect whether a video (a short-form video) is a clip of another (a long-form video), outperforming existing solutions, and implemented a concurrent pool management tool to fully utilize a multicore cloud server using Javascript.}
        \resumeItem{Developed a stateless interactive rating and data collection system with CI/CD integration using GitHub Actions and Cloudflare Workers for data storage, enabling users to label the political leaning of long and short videos based on transcripts and some pairing information.}
        \resumeItem{Assessed methods for measuring the leanings of media houses, utilizing AllSides and other existing models.}
      \resumeItemListEnd

    \resumeSubheading
      {Researcher}{April 2024 -- now}
      {Independent}{}
      \resumeItemListStart
        \resumeItem{Worked with Jimmy Koppel on comparing and analyzing the advantages and disadvatanges of using tutorials over the LLM system in helping experienced programmers in coding, especially in learning new, unfamiliar material.}
        \resumeItem{Discuss, design, and implement an analysis plan to code data and write a paper right now.}
      \resumeItemListEnd

    \resumeSubheading
      {HCI Research Course Project}{September 2023 -- December 2023}
      {University of California San Diego}{}
      \resumeItemListStart
        \resumeItem{Investigated whether an avatar appearing in the ChatGPT interface would help create a better communication environment between humans and machines, particularly in terms of alleviating loneliness. Added an avatar to the ChatGPT interface and found no significant difference in a 10 people pilot study}
      \resumeItemListEnd

    \resumeSubheading
      {Graduate Student Researcher}{January 2023 -- September 2023}
      {University of California San Diego}{}
      \resumeItemListStart
        \resumeItem{Worked with Michael Coblenz on the usability analysis of autocomplete. Paper published at FSE 2024.}
        \resumeItem{Designed and executed an experiment with 32 participants using an eye tracker to assess the costs and benefits of IDE-based autocomplete for programmers working with an unfamiliar API. Implemented algorithms from prior research in custom-built analysis scripts and used JMP to analyze the data. Performed open-coded analysis.}
        \resumeItem{Found that participants who used autocomplete learned more while spending less time reading documentation, although autocomplete did not significantly reduce the number of keystrokes required to complete tasks.}
        \resumeItem{Developed an user interaction tracking platform for VSCode for the web, integrating Tobii Eye Tracker 5 to capture eye movement and user interactions across in-editor events, with a JavaScript frontend and C\# backend; reduced frontend-backend communication latency to one-tenth of the existing solution, tested over 80 hours with 33 users.}
      \resumeItemListEnd

    \resumeSubheading
      {Usability of PL Course Project}{September 2022 -- December 2022}
      {University of California San Diego}{}
      \resumeItemListStart
        \resumeItem{Investigated whether Python Type Hints are helpful for competitive programmers.}
        \resumeItem{Designed and executed an experiment, containing tasks and interview, with 4 participants.}
        \resumeItem{Found that Python type hints were not significantly more useful for competitive programmers in terms of task completion time and debugging time. Open coded of interview results and transcripts indicated programmers feel adding type hints and the popped out autocomplete suggestions to be annoying.}
      \resumeItemListEnd
  \resumeSubHeadingListEnd
