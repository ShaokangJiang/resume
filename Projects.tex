\resumeSubHeadingListStart
  \resumeProjectHeading
    {\textbf{HealthCare Chatbot} $|$ \emph{Expo, Worker, GitHub, Jest, Vitest, Figma}}{2024}
    \resumeItemListStart
      \resumeItem{Led a team of five to create a chatbot with CI/CD pipelines using Agile methodologies. Self-delegate leadership roles when necessary. Consulted with experts in health care to understand their needs and requirements.}
\resumeItem{Developed a chatbot using Expo to provide healthcare information and answer questions via a fine-turned GPT service. Backend services were built with JavaScript and deployed on Cloudflare Workers to handle user information management, session storage,  authentication and token management, using salted hashes for password storage.}
\resumeItem{Learned methods for managing disputes and mediation conflicts within a team, including documenting divergence points and finding common ground. Realized the importance of communication in a team.}
\resumeItem{Integrated Material Design principles and conducted unit tests using Jest for the Expo app and Vitest for the middleware server. Utilized GitHub Actions for the CI/CD pipeline.}
    \resumeItemListEnd

\resumeProjectHeading
    {\textbf{Eyetracker Monitoring Platform} $|$ \emph{C\#, JS, Tampermonkey, Chrome, WebSockets, VSCode}}{2023, 2024}
    \resumeItemListStart
      \resumeItem{Developed a universal user interaction tracking platform that includes eye movement tracking during coding in VSCode. Integrated the Tobii Eye Tracker 5 to link eye movement data with in-editor events while ensuring compliance with legal requirements.}
\resumeItem{The 2023 edition tracks specific window appearances and user interactions in the web-based version of VSCode. The frontend is built with JavaScript and deployed as a Chrome extension, while the backend is developed in C\#. Both ends communicate using Chrome native messaging. The system successfully ran for 80 hours of user testing with 33 different users without any issues.}
\resumeItem{The 2024 edition tracks the entire window arrangement and user interactions in VSCode, with the ability to define and track additional events as needed. The frontend is built with JavaScript and deployed as a script in VSCode, while the backend is developed in C\#. Both ends communicate using WebSockets.}
    \resumeItemListEnd

\resumeProjectHeading
    {\textbf{Financial Tutoring website and app} $|$ \emph{HTML, Embedded JavaScript, Worker, JavaScript, echarts, Expo, NLP.js}}{2020}
    \resumeItemListStart
      \resumeItem{Designed and led a team to develop a static financial education website for Capital One. The site includes interactive learning tools, such as quizzes and a self-built, stateless learning progress tracking system. It also has various utilities, including a yearly budgeting tool.}
\resumeItem{Built front end using embedded JavaScript for the main site and HTML with JavaScript for utility tools. Developed backend with JavaScript and deployed on Cloudflare Workers to handle requests, including the encryption and decryption of learning stage information using AES and crypto and a dynamically rendered quiz system. Users can download their encrypted learning progress and upload it to decode and resume learning, or use the same browser to continue learning. No data is stored on the server.}
\resumeItem{Created detailed implementation guidance for group members and a user manual for the website and learning content. Integrated everything to automatically build and deploy using GitHub Actions.}
\resumeItem{Developed an app to help individuals organize expenses and income using voice interaction and simple chat, utilizing a local natural language understanding engine for Capital One.}
\resumeItem{Built with Expo, packaged, and tested for both Android and iOS, the app enables users to sync data between devices via JSON import/export, with all data stored locally for privacy.}
\resumeItem{Followed Agile methodologies and communicated with stakeholders to understand their needs.}
\resumeItem{Interacted with mentors from Capital One to discuss and understand the software development stages.}
    \resumeItemListEnd
\resumeSubHeadingListEnd