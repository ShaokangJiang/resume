
\section{Research Experience}

\outerlist{

\entryexp
	{\textbf{Graduate student researcher} | University of California San Diego}{Jan 2023\textendash Sept 2023}
\innerlist{
	\entry{Worked with Michael Coblenz on the usability analysis of autocomplete}
	\entry{Designed and executed an experiment with 32 participants using an eye tracker to evaluate the costs and benefits of IDE-based autocomplete features to programmers who use an unfamiliar API; analyzed data using JMP; and wrote a paper for the study.}
	\entry{Found that participants who used autocomplete learned more about the API while spending less time reading the documentation; Found autocomplete did not significantly reduce the number of keystrokes required to finish tasks}
	% \entry{Found the primary benefit of autocomplete is in providing information, not in reducing time spent typing.}
	\entry{Acquired fundamental skills in conducting empirical research studies and learned various methods for handling and understanding the implications of eye-tracking data.}
	\entry{Successfully designed an experiment with well-formed hypotheses and a comprehensive knowledge test suite to accurately capture the benefits and costs of autocomplete. Additionally, designed procedures and produced scripts to analyze data from various sources, including a substantial amount of data from a consumer-level eye tracker.}
}
% What is the hardest thing
% Addressed problems by xxx
% What i did and result

% Many IDEs support an autocomplete feature, which may increase developer productivity by reducing typing
% requirements and by providing convenient access to relevant information. However, to date, there has been
% no evaluation of the actual benefit of autocomplete to programmers. We conducted a between-subjects
% experiment (N=32) using an eye tracker to evaluate the costs and benefits of IDE-based autocomplete features
% to programmers who use an unfamiliar API. Participants who used autocomplete spent significantly less time
% reading documentation and got significantly higher scores on our post-study API knowledge test, indicating
% that it helped them learn more about the API. However, autocomplete did not significantly reduce the number
% of keystrokes required to finish tasks. We conclude that the primary benefit of autocomplete is in providing
% information, not in reducing time spent typing.

\entryexp
	{\textbf{Usability of PL course project} | University of California San Diego}{Sept 2022\textendash Dec 2022}
\innerlist{
	\entry{Worked on investigating whether Python Type Hints are helpful for competitive programmers(CP).}
	\entry{Designed and executed the experiment with 4 participants;}
	\entry{Did not find Python type hints to be significantly more useful for competitive programming in terms of task completion time and debugging time. Survey results indicate that programmers also dislike type hints and autocomplete suggestions because they are annoying.}
}

\entryexp
	{\textbf{HCI research course project} | University of California San Diego}{Sept 2023\textendash Dec 2023}
\innerlist{
	\entry{Investigating whether an avatar appearing in the ChatGPT interface would help create a better communication environment between humans and machines, particularly in terms of alleviating loneliness.}
}

\entryexp
	{\textbf{Research Assistant} | Illinois Institute of Technology}{Jan 2019\textendash May 2019}
\innerlist{
	\entry{Implemented POW(Bitcoin) blockchain and POS(EOSIO) blockchain system on a blockchain emulator to help them measure the performance of the emulator}
}

}
